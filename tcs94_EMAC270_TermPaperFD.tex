\documentclass[12pt]{article}

% Math and symbol packages
\usepackage{amsmath}
\usepackage{amssymb}
\usepackage{derivative}
\usepackage{gensymb}

% Figure Packages
\usepackage{graphicx}
\usepackage{wrapfig}
\usepackage{epstopdf}
\usepackage{float}
\usepackage{subfigure}
\usepackage{caption}

% Formatting
\usepackage{inputenc}
\usepackage[left=2.54cm,right=2.54cm,top=2.54cm,bottom=2.54cm]{geometry}
\usepackage[doi=true]{achemso}
\usepackage[numbers,sort&compress]{natbib}
\usepackage{setspace}
\usepackage{multicol}
\usepackage{titlesec}

% Header and indent packages
\usepackage{fancyhdr}
\usepackage{indentfirst}

% Create Title Section
\title{\textbf{Polymers for Extreme Temperature Applications}}
\author{Trevor Swan \\
Department of Macromolecular Science and Engineering \\ 
Case Western Reserve University \\
Cleveland, OH 44016-7079}
\date{October 23, 2024}

\begin{document}

\maketitle

\hrule height 1pt
\vspace{1em}

\noindent
\begin{minipage}[t]{0.2\textwidth}
    \subsection*{Article Info}
    \vspace{-0.5em}
    \hrule height 1pt
    \vspace{0.5em}
    \textit{Keywords:}\\
    	Hypersonic\\
        Poly($\rho$-phenylene terephthalamide)\\
        Polydopamine\\
    	Morphology
\end{minipage}%
\hfill
\begin{minipage}[t]{0.75\textwidth}
    \section*{Abstract}
    \vspace{-0.97em}
    \hrule height 1pt
    \vspace{0.5em}
    Aromatic polyamide fibers are of increasing interest for extreme temperature polymer and composite applications, offering  low thermal conductivity and malleable mechanical properties in composites. Poly($\rho$-phenylene terephthalamide) (Kevlar$^{\text{\textcopyright}}$) fibers are mechanically weak in composites though they are thermally insulating. To create a more apt composite for extreme environment applications, we can treat poly($\rho$-phenylene terephthalamide) with polydopamine, aminated carbon nanotubes, or ethylene glycol diglycidyl ether. These treatments improve fiber-matrix interactions of the composite, allowing the composite is able to maintain its thermal properties while also inheriting the mechanical stability of the resin. Treated poly($\rho$-phenylene terephthalamide) fibers demonstrate viability for extreme-temperature applications, degrading above 500 \degree C while maintaining high modulus and tensile strength in mechanical tests. These findings are crucial for developing polymers and composites for hypersonic travel, where materials are subjected to high air resistance, friction, and temperatures.
    
\end{minipage}

\vspace{1em}
\hrule height 1pt

\begin{multicols}{2}

\section{Introduction}

\indent Hypersonic travel has long been a major point of interest in the aerospace industry and scientific community alike. Speeds exceeding Mach 5, or about 6,200 km/h, offer significant potential with respect to both commercial and defense based applications. Since the 1980s, NASA and the United States Government have sought state-of-the-art high-performance composites for applications like High Speed Civil Transport (HSCT), air-to-air tactical missiles, and Advanced Technical Fighters (ATFs). \citep{Lau2014} However, the extreme temperature conditions associated with hypersonic travel pose substantial material challenges. Materials used must exhibit high thermal resistance while also being manufacturable and moldable into complex shapes for a wide range of applications. Moreover, they must have consistent mechanical properties across a wide temperature range. Finding a material that meets these demanding criteria has proven to be remarkably difficult.
	
\indent Due to the versatility and wide range of properties that polymers offer, polymer science has been at the heart of addressing these material concerns. Unlike metals and other commonly used materials, polymers have the potential to be incredibly lightweight solutions while also having tunable properties for specific applications. These properties are crucial for hypersonic travel, enhancing fuel efficiency and maximizing carrying capacity. Through the study of polymers, optimized materials can be developed to withstand the extreme thermal and mechanical stresses encountered when traveling at hypersonic speeds. 

\indent Recent advancements in materials science and machine learning have enabled better predictions of properties and synthesis of high performance polymers and composites. As computer science has continued to evolve, researchers have been able to develop material solutions using finely tuned machine learning models, enhancing experimental efficiency by reducing trial-and-error. This review aims to analyze recent studies and advancements regarding the synthesis and application of high-performance polymers. Specifically, it aims to discuss those that can withstand temperatures exceeding 500 \degree C and their potential to revolutionize materials for use in extreme environments, such as hypersonic travel.
	
\section{Results and Discussion}

\subsection{Thermally Stable Polymeric Lubricants}

\indent Recent research has highlighted tribological solutions for extreme temperatures ranging from $-196$\degree C to $300$\degree C. Although this temperature range is below that of what is expected at hypersonic speeds, the tribological concepts studied here provide valuable insight for future materials operating above 500 \degree C. Also, they do offer potential coating and lubricant solutions for aircraft components that may be within this reduced temperature range. Aromatic thermosetting co-polyesters (ATSP) based coatings were tested in high friction environments by sliding them against metal counterparts. Upon this sliding, a transfer layer was created between the coating and the metal. This resulted in a low coefficient of friction (COF) with minimal wear of the ATSP coating. Along with being in the aforementioned temperature range, the coatings were also exposed to pressures up to 1 MPa while still experiencing negligible wear. \citep{Bashandeh2021} This pressure simulates the operational stresses in aerospace environments, illustrating the mechanical effectiveness of the coating. Being able to make use of these so-called 'self-lubricating' polymers is ideal in scenarios where the use of liquid lubricants is not feasible. Notably, ensuring a low COF with little material wear allows for longer term use of the coatings, which reduces maintenance costs and is therefore financially desirable.

\begin{figure}[H]
    \centering
    \includegraphics[width=0.45\textwidth]{Resources/References/Images/Bashandeh_Results_Summary.jpg}
    \caption{\scriptsize{Adapted from Bashandeh et al. 2021 \citep{Bashandeh2021}}}
    \label{fig:Bashandeh-Summary}
\end{figure}

\indent In addition to their study on Bashandeh et al. also report on polyether ether ketone (PEEK)-, polytetrafluoroethylene (PTFE), and Fluoropolymer (FP)-based coatings. As supported by Figure \ref{fig:Bashandeh-Summary} which summarizes the results of their work, these materials exhibit lower wear resistance and COF while being exposed to the same conditions as the ATSP coatings. To further analyze ATSP coatings, Scanning Electron Microscopy (SEM) was used to observe mechanical properties and behavior. While ATSP-based coatings show promise for hypersonic travel at first glance, it should be noted that the SEM results reveal microcracks forming around 300\degree C. Furthermore, ATSP on its own has a relatively low $T_g$ between $233$ and $244$\degree C. Considering the potential temperatures faced at hypersonic speeds as well as inherent mechanical stresses applied by the aircraft, these coatings may not be of much use. That being said, further refinement of the coating's thermal and mechanical properties  may make them being more suited for extreme temperature applications. 

\subsection{Machine Learning to Generate Compounds}

\indent In light of the shortcomings of the previously discussed coatings, there has been an effort to synthesize more specific materials with finely tuned material and thermal properties. Deriving synthesis schemes for these materials has recently been streamlined with advancements in machine learning. Using sets of data with known properties and deep neural networks, a variation auto-encoder (VAE) is able to construct desired output sequences to learn about specific polymer properties. \citep{Batra2020} Making use of the VAE model, Batra set property goals to $T_g>600K$ to generate a variety of possibilities for the polymer. Using its training data, the VAE was able to generate over 300 candidates to fit the input requirements. Specifically, some the high $T_g$ materials generated can be seen in Figure \ref{fig:Batra_Results}, and it should be noted that almost all of the $T_g$-fulfilling materials contain aromatic rings. This observation is important to keep in mind when evaluating candidates for high temperature applications, as polymers with higher glass transition temperatures are desirable for certain hypersonic aircraft components. Although not included below, Batra used the VAE to generate electrically stable materials as well, highlighting the effectiveness of machine learning in the development of highly specialized materials.

\begin{figure}[H]
    \centering
    \includegraphics[width=0.4\textwidth]{Resources/References/Images/Batra_Results.jpeg}
    \caption{\scriptsize{Adapted from Batra et al. 2020 \citep{Batra2020}}}
    \label{fig:Batra_Results}
\end{figure}

\indent The results of Figure \ref{fig:Batra_Results} are supported by a 1969 entry by F.E Arnold, which suggested that the thermal properties of polymerized aromatic compounds are desirable. \citep{Arnold1969} This indicates that the previously discussed study was on the right track, but that other similar compounds should be investigated for applications in more extreme environments.

\subsection{PPTA Fiber Morphology}

\vspace{-0.4em}

\indent Looking for aromatic based polymer backbones, research has pointed to Poly($\rho$-phenylene terephthalamide) (PPTA), a para-aramid polymer commonly referred to as Kevlar$^{\text{\textcopyright}}$, as a solution. \citep{Liu2019} These fibers have been long used in the aerospace and automotive industries as reinforced long fiber composites for their extremely high modulus and tensile strength. \cite{Cheng2005} The high modulus possessed by PPTA fibers is due to the high stiffness of the aromatic polyamide chains in conjunction with the largely distributed hydrogen bonding regions in the material. \citep{Wang2020} The strength of PPTA prepared composites, however, lacks in performance and strength due to poor interfacial adhesion between the fiber and the matrix. \citep{Kanbargi2017} Kanbargi et al. \citep{Kanbargi2017} argue that these composites can be made more viable by altering PPTA fiber surface morphology such that there can be covalent bonding between the altered fiber and the matrix. Kanbargi et al. found that mechanical and microwave pretreatments of the fiber allowed interactions with a coupling agent which altered the morphology of the fibers in such a way that the fiber-matrix adhesion was significantly increased. They made use of the coupling agent supercritical carbon dioxide (scCO$_2$) and found the interfacial adhesion to be up to 2 times greater than the control group. \citep{Kanbargi2017} The effect of the pretreatments on adhesion can be seen in more detail in Figure \ref{fig:Kanbargi_Results}. The results of their study is promising, as these composites could theoretically maintain both their low thermal conductivity and high mechanical strength upon altering their morphology.

\begin{figure}[H]
    \centering
    \includegraphics[width=0.4\textwidth]{Resources/References/Images/Kanbargi_pretreat_results.jpg}
    \caption{\scriptsize{Adapted from Kanbargi et al. 2017 \citep{Kanbargi2017}}}
    \label{fig:Kanbargi_Results}
\end{figure}

\subsection{Polydopamine Based Solutions}

\indent In an effort to increase PPTA's adhesive ability, researchers looked to nature for naturally occurring sticky substances. Marine mussels have proteins which contribute to their sticking ability, namely 3,4-dihydroxyphenylalanine ($L$-DOPA) and lysine. \citep{Waite2001} Lee et al. \citep{Lee2007} found that dopamine, a molecule with both the properties of $L$-DOPA and lysine, can self-polymerize to form polydopamine (PDA) film to be used on almost all substances. \citep{Yuan2017} Polydopamine is frequently used to modify fibers because of its hydrophilicity and stable polymerization properties under mild reaction conditions. The benefits of polydopamine film was directly observed when testing nitrile butadiene rubber (NBR) composites. \citep{Kong2018}. Kong et al. \citep{Kong2018} found that NBR/PDA-PPTA composites experienced considerably higher tensile strengths as the DOPA concentration increased. This is directly indicative of strong interfacial adhesion between the fiber and matrix. As seen in Figure \ref{fig:Kong_Results}, there is a clear correlation between PDA (DOPA) concentration and the tensile strength of the composite material. This can be attributed to the introduction of PDA, which adds a layer on the fiber which becomes thicker and more coarse as the amount of PDA added increases. The results of Kong et al's \citep{Kong2018} study illustrates the effectiveness of PDA coatings in modifying the surface of PPTA fibers, thereby increasing the interfacial adhesion between them and their respective matrix. This allows the mechanical properties of the composites to be controlled and refined while also maintaining the thermal stability of the PPTA fibers. 

\begin{figure}[H]
    \centering
    \includegraphics[width=0.4\textwidth]{Resources/References/Images/Kong_Results.jpg}
    \caption{\scriptsize{Adapted from Kong et al. 2008 \citep{Kong2018}}}
    \label{fig:Kong_Results}
\end{figure}

\indent Due to the increasing price of dopamine over the years, researches also explored using ethylene glycol diglycidyl ether (EGDE) and poly(catechol/polyamine) (PCPA) based solutions to enhance PPTA's poor interfacial adhesion. \citep{LeiWang2017} These solutions are more affordable while still maintaining the previously discussed thermal and mechanical properties. As supported by Figure \ref{fig:LeiWang_Results}, the PPTA-PCPA-EDGE fibers performed much better in the long run, while the PPTA-PCPA and PPTA fiber showed reduced weight loss at approximately 500 \degree C. Specifically, the PPTA-PCPA-EDGE fibers experience about a 10wt\% drop in residual weight at theoretical hypersonic temperatures.

\begin{figure}[H]
    \centering
    \includegraphics[width=0.4\textwidth]{Resources/References/Images/LeiWang_Results.jpg}
    \caption{\scriptsize{Adapted from Lei Wang et al. 2017 \citep{LeiWang2017}}}
    \label{fig:LeiWang_Results}
\end{figure}

\indent That being said, the use of EGDE in the PPTA fibers does allow the surface morphology of the fibers to be altered, enhancing their ability to form stronger polymer composites. The strong chemical bonds formed between the PPTA fibers grafted with EGDE and the rubber significantly enhance fiber durability and stress transfer efficiency. \citep{Kong2018} Chemically, the increased grafting time allows the introduction of epoxy groups which can be reacted with sulfur radicals generated during the vulcanization process of the rubber. \citep{Zhang2012} As discussed later in this review, there are solutions to PPTA's interfacial adhesion problem that perform better across the board, but this experiment does show some promise especially with more refined materials for more desired properties.

\subsection{Carbon Nanotube Grafting}

\indent Researchers also made use of the adhesive strength of polydopamine with aminated carbon nanotubes (NH$_2$-CNTs) to enhance the interfacial adhesion between PPTA fibers and a rubber matrix. Yang et al. \citep{Yang2019} used PDA as a precursor coating for PPTA fibers so that they could then be grafted with the CNTs mentioned previously. The group performed thermogravimetric analysis on 6 samples to compare the weight loss over time. In Figure \ref{fig:Yang_Results_1}, A-PPTA represents as-is PPTA fibers, while PDA-PPTA represents PPTA treated only with PDA. PPTA-1, PPTA-2, and PPTA-3 are represent the different grafted PPTA fibers in order of increasing reaction time. The results of their analysis expectedly revealed that the duration of reaction with the aminated carbon nanotubes improved the thermal properties of the PPTA fibers. In other words, grafting extremely thermally insulating materials like CNTs into the PPTA fibers results in a better thermal properties overall. Figure \ref{fig:Yang_Results_1} shows that the PPTA with the largest amount of grafted CNTs has the highest weight retention across the TGA range. Specifically, this fiber was able to withstand temperatures around 500\degree C with minimal weight loss. 

\begin{figure}[H]
    \centering
    \includegraphics[width=0.4\textwidth]{Resources/References/Images/Yang_Results_1.png}
    \caption{\scriptsize{Adapted from Yang et al. 2019 \citep{Yang2019}}}
    \label{fig:Yang_Results_1}
\end{figure}

\indent Using PDA as a precursor for the PPTA fibers allows for increased tensile strength in the resulting composites, addressing the primary concern when working with PPTA fibers. Yang et al. \citep{Yang2019} measured the pull-out force from the various PPTA sample with the rubber. As seen in Figure \ref{fig:Yang_Results_2}, the pull out force increased from the A-PPTA sample to the PPTA-3 sample, increasing from $28.3$N to $45.2$N \citep{Yang2019}. This trend is again supported by the overwhelmingly smooth surface on the A-PPTA fibers which prohibits strong interfacial adhesion between the fiber and rubber matrix. As the reaction time increases, more NH$_2$-CNTs can be deposited onto the fiber which increases the roughness of the fiber surface. This promotes surface energy of the fibers \citep{Yang2019}, which correlates with an increase in interactions between the PPTA fibers and rubber matrix. This suggests that the addition of NH$_2$-CNTs is beneficial to not only the thermal properties of produced composites with PPTA, but also for increasing the mechanical properties which were previously lacking among conventional PPTA fibers.

\begin{figure}[H]
    \centering
    \includegraphics[width=0.4\textwidth]{Resources/References/Images/Yang_Results_2.png}
    \caption{\scriptsize{Adapted from Yang et al. 2019 \citep{Yang2019}}}
    \label{fig:Yang_Results_2}
\end{figure}

\subsection{PDA Dip-Coated Carbon Fibers}

\indent In an attempt to find another carbon-based solution, carbon fibers (CFs) were studied as they have excellent mechanical properties and low weight. However, issues arise when using them to reinforce composites due to the low surface activity of untreated carbon fibers. Chen et al. \citep{ShushengChen2014} attempted to counteract this through oxidation methods, fiber sizing, and coating. Oxidation methods proved to not be effective as they damaged the mechanical properties of the fibers, and fiber sizing proved ineffective as most sizing agents are not able to interact with the hydrophobic CFs. Making use of dip-coating the CFs in polydopamine film, the CFs were converted from hydrophobic to hydrophilic. \citep{ShushengChen2014} The group then performed thermogravimetric analysis with short carbon fibers (SCFs) treated with the aforementioned dip coating method. This analysis revealed that the treated SCFs, referred to hereafter as PDA-SCFs, started to lose weight around 300\degree C, specifically losing about 2wt\% at 500\degree C. \citep{ShushengChen2014} To evaluate whether the surface activity of the fibers had improved, they were then added in 15wt\% to both polar and non-polar resins. Upon analysis, the tensile strength of the composites increased by 70 and 60\% respectively, which was about a 1.5- to 2-fold improvement. \citep{ShushengChen2014} The results of Chen et al.'s \citep{ShushengChen2014} study illustrates the effectiveness of polydopamine in creating functional fibers from otherwise unusable ones like CFs for specific composites involving various resins and matrices. 

\section{Conclusions}

\indent In this review, I have discussed the potential polymeric solutions for high-temperature and hypersonic applications. Namely, poly($\rho$-phenylene terephthalamide), more commonly referred to as PPTA or Kevlar$^{\text{\textcopyright}}$, fiber can be altered to obtain desirable mechanical properties while maintaining low thermal conductivity. \citep{LiTong2022} PPTA fiber alone own is not suitable for high-temperature applications due to its poor mechanical stability in composites, and therefore it must be enhanced experimentally. This is necessary to prevent mechanical wear from high amounts of stress applied by moving mechanical parts, air resistance, and pressure variations. Given the nature of polymers, this task is not arduous. Through the use of many different experimental techniques, these properties can be closely monitored as modifications are made. 

\indent By making use of polydopamine \citep{Yuan2017}, ethylene glycol diglycidyl ether (EGDE)\citep{LeiWang2017}, altering the polymers morphology \citep{Kanbargi2017}, introducing nanocomposites \citep{BoZhang2021}, and utilizing carbon nanotubes \citep{Yang2019}, PPTA can become more viable for hypersonic travel. Enhancing this material's mechanical properties allows it to withstand high mechanical stresses and frictional forces while also maintaining its already resent thermal properties. The altered fibers can withstand temperature ranges of 30\degree C to 800\degree C with reduced thermal weight loss \citep{ShushengChen2014}. While there is room for improvement and potentially undiscovered solutions, PPTA remains a feasible and widely produced solution to a hot topic in the polymer community. Through experimental alterations, PPTA-utilizing composite materials prove to be a useful tool for designing materials fit for hypersonic travel, paving the way for advancements in high-performance polymer applications.

\end{multicols}

\newpage 

\bibliographystyle{achemso}
\bibliography{refs} 

\end{document}
