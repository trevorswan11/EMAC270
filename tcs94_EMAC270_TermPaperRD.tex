\documentclass[12pt]{article}

% Math and symbol packages
\usepackage{amsmath}
\usepackage{amssymb}
\usepackage{derivative}
\usepackage{gensymb}

% Figure Packages
\usepackage{graphicx}
\usepackage{wrapfig}
\usepackage{epstopdf}
\usepackage{float}
\usepackage{subfigure}

% Formatting and Random Text Generation
\usepackage{inputenc}
\usepackage[left=2.54cm,right=2.54cm,top=2.54cm,bottom=2.54cm]{geometry}
\usepackage{lipsum}
\usepackage{achemso}  % Load ACS style using achemso
\usepackage[numbers,sort&compress]{natbib}  % For numerical citations

% Header and indent packages
\usepackage{fancyhdr}
\usepackage{indentfirst}

% Create Title Section
\title{Polymers for Extreme Temperature Applications}
\author{Trevor Swan \\
Department of Macromolecular Science and Engineering \\ 
Case Western Reserve University \\
Cleveland, OH 44016-7079}
\date{}

\begin{document}

\maketitle

\section{Abstract}
	I will write the abstract as the last part of my paper, as it should survey the topics and results discussed. Citation examples for LaTeX, which I will be using to write and format this paper, compilation. I am using ACS as a reference style. Also, I have appropriately removed the outline from this document.
\indent I am using BibTex to compile my refs.bib document for this paper, please let me know if my citations are done incorrectly so I can change them sooner rather than later.
\section{Introduction}
	\begin{enumerate}
		\item Desire for hypersonic travel commercially and narrower scopes
		\item Difficulty finding polymers to withstand extreme temperature scenarios
		\item High friction scenarios both with physical and gases (air resistance) need to be accounted for
		\item Space stations can make use of temp withstanding polymers in either direction
		\item Main concern with hypersonic travel is the lack of ability to handle friction with air
		\item Explicitly state the purpose of these studies
	\end{enumerate}
\indent Hypersonic travel has long been a major point of interest in the aerospace industry and scientific community alike. Speeds exceeding Mach 5 offer significant potential with respect to both commercial and defense based applications. However, the extreme temperature conditions associated with hypersonic travel pose substantial material challenges. Materials used must exhibit high thermal resistance while also being manufacturable and moldable into various complex shapes for a wide range of applications. Finding a material that meets these demanding criteria has proven to be remarkably difficult.
	
\indent Polymer science has been at the heart of addressing these material concerns due to the versatility and wide range of properties that polymers offer. Unlike metals and other commonly used materials, polymers have the potential to be incredibly lightweight solutions while also having tunable properties for specific applications. Through the study of polymers, optimized materials can be derived such that they can withstand the extreme thermal and mechanical stresses encountered when traveling at hypersonic speeds. 

\indent \textit{Discuss recent research and discuss what topics the review article will talk about(ATSP-\cite{Bashandeh2021}, PPTA\cite{LiTong2022}), implying a vertical vs. horizontal approach}
	
\section{Results and Discussion}
	\begin{enumerate}
		\item Polymer coatings can mitigate low thermal resistance
		\item Polyimides very thermo-oxidatively resistant
		\item Discuss synthesis of polyimides and behavior at temperatures above 500\degree C
		\item Thermoplastic vs. thermosey polyimides 
		\item Nanoparticles: their use and benefit to making travel like this possible
	\end{enumerate}
	
\section{Conclusions}
	Restate, succinctly, the results section.
	\begin{enumerate}
		\item Discuss without logical sentences
		\item Discuss possible grievances and comments
		\item Discuss future solutions and ideas
	\end{enumerate}	 

\bibliographystyle{achemso}
\bibliography{refs} 

\end{document}