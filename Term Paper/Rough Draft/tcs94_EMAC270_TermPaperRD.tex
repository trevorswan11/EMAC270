\documentclass[12pt]{article}

% Math and symbol packages
\usepackage{amsmath}
\usepackage{amssymb}
\usepackage{derivative}
\usepackage{gensymb}

% Figure Packages
\usepackage{graphicx}
\usepackage{wrapfig}
\usepackage{epstopdf}
\usepackage{float}
\usepackage{subfigure}

% Formatting and Random Text Generation
\usepackage{inputenc}
\usepackage[left=2.54cm,right=2.54cm,top=2.54cm,bottom=2.54cm]{geometry}
\usepackage{lipsum}
\usepackage[doi=true]{achemso}  % Load ACS style using achemso
\usepackage[numbers,sort&compress]{natbib}  % For numerical citations

% Header and indent packages
\usepackage{fancyhdr}
\usepackage{indentfirst}

% Create Title Section
\title{Polymers for Extreme Temperature Applications}
\author{Trevor Swan \\
Department of Macromolecular Science and Engineering \\ 
Case Western Reserve University \\
Cleveland, OH 44016-7079}
\date{}

\begin{document}

\maketitle

\section{Abstract}

\indent I will write the abstract as the last part of my paper, as it should survey the topics and results discussed. Citation examples for LaTeX, which I will be using to write and format this paper, compilation. I am using ACS as a reference style. Also, I have appropriately removed the outline from this document.
	
\indent I am using BibTex to compile my refs.bib document for this paper, please let me know if my citations are done incorrectly so I can change them sooner rather than later. \textbf{Please note that I have more references than listed here, but the paragraphs I wrote for this draft do not make use of all of them, so LaTeX doesn't include them}.

\indent As i am not writing an abstract until the rest of the paper is complete, I included more detail in the next 2 sections, though I have a lot of work to do. I also plan on adding images to the paper down the line.

\section{Introduction}

\indent Hypersonic travel has long been a major point of interest in the aerospace industry and scientific community alike. Speeds exceeding Mach 5 offer significant potential with respect to both commercial and defense based applications. However, the extreme temperature conditions associated with hypersonic travel pose substantial material challenges. Materials used must exhibit high thermal resistance while also being manufacturable and moldable into various complex shapes for a wide range of applications. Moreover, they must have consistent mechanical properties across a wide temperature range. Finding a material that meets these demanding criteria has proven to be remarkably difficult.
	
\indent Polymer science has been at the heart of addressing these material concerns due to the versatility and wide range of properties that polymers offer. Unlike metals and other commonly used materials, polymers have the potential to be incredibly lightweight solutions while also having tunable properties for specific applications. Through the study of polymers, optimized materials can be derived such that they can withstand the extreme thermal and mechanical stresses encountered when traveling at hypersonic speeds. 

\indent Within the past decade, scientists have been looking into solutions to address these concerns. Recent advancements in materials science and machine learning have enabled better predictions of properties and synthesis of high performance polymers and composites. It is the purpose of this review to analyze these recent studies and advancements and discuss their significance in revolutionizing the field of hypersonic travel.
	
\section{Results and Discussion}

\indent Recent research has highlighted tribological solutions for extreme temperatures ranging from $-196$\degree C to $300$\degree C. While these temperatures do not reach the $500+$\degree C expected with hypersonic travel, they do offer coating and lubricant solutions aircraft components that may be within this range. Aromatic thermosetting co-polyesters (ATSP) based coatings were tested in high friction environments by sliding them against metal counterparts. Upon this sliding, a transfer layer was created between the coating and the metal. This resulted in a low coefficient of friction (COF) while the ATSP wear also showed minimal wear. Along with being in the aforementioned temperature range, the coatings were also exposed to pressures up to 1 MPa while still experiencing negligible wear. \citep{Bashandeh2021} Being able to make use of these so-called 'self-lubricating' polymers is ideal in scenarios where the use of liquid lubricants is not feasible. Notably, ensuring a low COF with little material wear allows for longer term use of the coatings, which reduces maintenance costs and is therefore financially desirable.

\begin{figure}[h]
    \centering
    \includegraphics[width=0.45\textwidth]{../Resources/References/Images/Bashandeh_Results_Summary.jpg}
    \caption{\scriptsize{Adapted from Bashandeh et al. 2021 \citep{Bashandeh2021}}}
    \label{fig:Bashandeh-Summary}
\end{figure}

\indent In addition to their study on Bashandeh et al. also report on polyether ether ketone (PEEK)-, polytetrafluoroethylene (PTFE), and Fluoropolymer (FP)-based coatings. As supported by Figure \ref{fig:Bashandeh-Summary} which summarizes the results of their work, these materials cannot withstand the same conditions as the ATSP-based coatings. To further analyze ATSP coatings, Scanning Electron Microscopy (SEM) was used to observe mechanical properties and behavior. While ATSP-based coatings show promise for hypersonic travelat a first glance, it should be noted that the SEM results reveal microcracks forming around 300\degree C, and that ATSP on its own has a relatively low $T_g$ between $233$ and $244$\degree C. Considering the potential temperatures faced at hypersonic speeds as well as inherent mechanical stresses applied by the aircraft, these coatings may not be of much use.That being said, upon further refining of the coatings, thermal and mechanical properties of these coatings may be more suited for extreme temperature applications. 

\indent Keeping in mind the shortcomings of the previously discussed coatings, there has been an effort to synthesize more specific materials with finely tuned material and thermal properties. Deriving synthesis schemes and the desired product, more broadly, for these materials has recently been streamlined with advancements in machine learning. Using sets of data with known properties and deep neural networks, a variation auto-encoder (VAE) is able to construct desired output sequences to learn about specific polymer properties. \citep{Batra2020} Making use of the VAE model, Batra set property goals to $T_g>600K$ to generate a variety of possibilities for the polymer. Using its training data, the VAE was able to output plenty of candidates (300+) to fit the input requirements. Specifically, some the high $T_g$ materials generated can be seen in Figure \ref{fig:Batra_Results}, and it should be noted that almost all of the $T_g$-fulfilling materials contain aromatic rings. This observation is important to keep in mind when evaluating candidates for high temperature applications, as polymers with higher glass transition temperatures are desirable for certain hypersonic aircraft components. Although not inlcluded below, Batra used the VAE to generate electrically stable materials as well, emphasizing the effectiveness of machine learning in the development of highly specialized materials.

\begin{figure}[h]
    \centering
    \includegraphics[width=0.4\textwidth]{../Resources/References/Images/Batra_Results.jpeg}
    \caption{\scriptsize{Adapted from Batra et al. 2020 \citep{Batra2020}}}
    \label{fig:Batra_Results}
\end{figure}

\indent Looking for aromatic based polymer backbones,  

\section{Conclusions}

\indent Poly($\rho$-phenylene terephthalamide), more commonly referred to as PPTA, fiber has desirable mechanical properties while maintaining low thermal conductivity.\citep{LiTong2022} PPTA fiber on its own is not suitable for high temperature applications and must be enhanced experimentally. Due to the nature of polymers, this is no arduous task. Through the use of Scanning Electron Microscopy (SEM), these properties can be closely monitored as modifications are made. 

\indent By making use of Ethylene glycol diglycidyl ether (EGDE)\citep{LeiWang2017}, altering the polymers morphology \citep{Kanbargi2017}, introducing nanocomposites \citep{BoZhang2021}, and utilizing carbon nanotubes \citep{Yang2019}, PPTA can become viable for hypersonic travel. Enhancing this materials already present thermal and mechanical properties allows it to withstand temperature ranges between 30\degree C and 800\degree C with decreased thermal weight loss \citep{ShushengChen2014}. While there is still room for improvement, composite materials utilizing PPTA fibers prove to be a useful tool for designing materials fit for hypersonic travel.

\bibliographystyle{achemso}
\bibliography{../refs} 

\end{document}
